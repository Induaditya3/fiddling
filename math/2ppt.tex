\documentclass{beamer}
\usetheme{CambridgeUS}

% Remove footers
\setbeamertemplate{footline}[frame number]

\title{Discrete Random Variable}
\author{Aditya Raj}
\institute{GKCIET, Malda \\ Computer Science and Engineering Department \\ Roll No: 35530824051}
\date{February, 2025}

\begin{document}

\begin{frame}
    \titlepage
\end{frame}

\begin{frame}
    \frametitle{Introduction}
    \textbf{What is a Random Variable?} \\
    A random variable (or stochastic variable) is a function which assigns a number to each point of a sample space. It is usually denoted by capital letters such as $X$ and $Y$. \\
    Mathematically, \\
    $$X: \Omega \rightarrow \mathbb{R}$$
    The set of values $x$ of random variable $X$ such that $P(X= x) > 0$ is called the 'support' of $X$.

    \textbf{Types of Random Variable}
    \begin{itemize}
        \item \textbf{Discrete Random Variable} - It takes a finite or countably infinite number of values.
        \item \textbf{Nondiscrete Random Variable} - It can take a noncountably infinite number of values.
    \end{itemize}
\end{frame}

\begin{frame}
    \frametitle{Example of Discrete Random Variable}
    Consider an experiment where we toss a fair coin twice. The sample space consists of four possible outcomes: $S = \{HH, HT, TH, TT\}$. Here are some random variables on this space:
    \begin{enumerate}
        \item Let $X$ be the number of Heads. Then,
        $$X(HH) = 2, \quad X(HT) = 1, \quad X(TH) = 1, \quad X(TT) = 0$$
        \item Let $Y$ be the number of Tails. In terms of $X$, $Y = 2 - X$. Or
        $$Y(HH) = 0, \quad Y(HT) = 1, \quad Y(TH) = 1, \quad Y(TT) = 2$$
        \item Let $I$ be 1 if the first toss lands Heads and 0 otherwise.
        $$I(HH) = 1, \quad I(HT) = 1, \quad I(TH) = 0, \quad I(TT) = 0$$
        This is an example of what is called an indicator random variable since it indicates whether the first toss lands Heads, using 1 to mean “yes” and 0 to mean “no”.
    \end{enumerate}
\end{frame}

\begin{frame}
    \frametitle{Distribution of Random Variable}
    The distribution of a random variable specifies the probabilities of all events associated with that random variable. For discrete random variables, we generally use the probability mass function (PMF).

    The probability mass function (PMF) of a discrete random variable $X$ is the function $p_X$ given by:
    $$p_X (x) = P (X = x)$$
    Note that this is positive if $x$ is in the support of $X$, and $0$ otherwise.

    Here, $X = x$ denotes an event, consisting of all outcomes $s$ to which $X$ assigns the number $x$. Formally,
    $${X = x} \equiv {s \in S : X(s) = x}$$
\end{frame}

\begin{frame}
    \frametitle{Example of PMF}
    Using the previous example of two coin tosses, let us find PMFs of random variables $X$, $Y$, and $I$.
    \begin{itemize}
        \item For $X$:
        \[p_X (0) = P (X = 0) = \frac{1}{4}, \quad p_X (1) = P (X = 1) = \frac{1}{2}, \quad p_X (2) = P (X = 2)  = \frac{1}{4}\]
        and $p_X (x) = 0$ for all other values of $x$.
        
        \item For $Y$:
        $$p_Y (0) = P (Y = 0) = \frac{1}{4}, \quad p_Y (1) = P (Y = 1) = \frac{1}{2}, \quad p_Y (2) = P (Y = 2) = \frac{1}{4}$$
        and $p_Y (y) = 0$ for all other values of $y$.
        
        \item For $I$:
        $$p_I (0) = P (I = 0) = \frac{1}{2}, \quad p_I (1) = P (I = 1) = \frac{1}{2}$$
        and $p_I (i) = 0$ for all other values of $i$.
    \end{itemize}
    % Link for inserted image can be added here
\end{frame}

\begin{frame}
    \frametitle{Valid PMFs}
    Let $X$ be a discrete random variable with support $x_1, x_2, \ldots$. The PMF $p_X$ of $X$ must satisfy the following two criteria:
    \begin{enumerate}
        \item \textbf{Nonnegative:} $p_X (x) > 0$ if $x = x_j$ for some $j$, and $p_X (x) = 0$ otherwise.
        \item \textbf{Sums to 1:} 
        $$\sum_{j=1}^{\infty} p_X(x_j) = 1$$
    \end{enumerate}
\end{frame}

\begin{frame}
    \frametitle{Conclusion}
    In summary, discrete random variables play a crucial role in probability theory and statistics. Understanding their properties, such as the probability mass function (PMF) and the support, is essential for analyzing random processes. 

    \textbf{Key Takeaways:}
    \begin{itemize}
        \item A random variable assigns numerical values to outcomes in a sample space.
        \item Discrete random variables can take on a countable number of values.
        \item The PMF provides a complete description of the distribution of a discrete random variable.
    \end{itemize}
\end{frame}

\begin{frame}
    \frametitle{References}
    \begin{enumerate}
        \item Joseph K. Blitzstein and Jessica Hwang, \textit{Introduction to Probability, Second Edition}
        \item Jean Walrand, \textit{Probability in Electrical Engineering and Computer Science: An Application-Driven Course}
    \end{enumerate}
\end{frame}

\end{document}